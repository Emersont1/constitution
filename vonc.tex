\subsection{Definition}
\begin{enumerate}[i.]
    \item If an elected committee member is unable to fulfil the duties mandated by their role, or if they refuse to do so, a Vote of No Confidence (VoNC) in that individual can take place.
\end{enumerate}

\subsection{Procedure}
A VoNC may take place according to one of the following procedures:
\begin{enumerate}[i.]
    \item Upon committee member proposal.
    \begin{enumerate}[1.]
        \item A committee member must propose the VoNC to the President or to the Union.
        \begin{enumerate}[(a).]
            \item The Union must be consulted, and an attempt to find a solution must be made before a valid VoNC may take place. All committee members have a right to be involved in this consultation.
            \item A valid VoNC may only take place through this procedure upon confirmation from the Union that the appropriate processes have been followed.
        \end{enumerate}
        \item The committee must then vote anonymously.
        \item A two-thirds majority is required for the VoNC to pass.
        \item If the VoNC is unsuccessful, the Committee Member in question may not be subjected to this procedure within the same University term.
    \end{enumerate}
    \item Upon petition by Society Members.
    \begin{enumerate}[1.]
        \item A petition must either be presented at a quorate General Meeting or to the Society President. The petition must be signed by either five percent of all Society Members or 20 Society Members, whichever is fewer.
        \item A vote of the Society Members must take place.
        \begin{enumerate}[(a).]
            \item The VoNC operates under the same quoracy and election guidelines as a General Meeting.
            \item The Committee Member in question shall have the right to present their response alongside the petitioned motion.
            \item A VoNC needs a simple majority to pass.
        \end{enumerate}
    \end{enumerate}
\end{enumerate}

\subsection{Result}
\begin{enumerate}[i.]
    \item If a Vote of No Confidence passes, then the individual to whom it pertains is removed from the committee, and, at the committee’s discretion, a by-election is called to fill their position.
\end{enumerate}

\subsection{Health or Personal Issues}
\begin{enumerate}[i.]
    \item If a committee member finds themselves unable to fulfil their duties due to health or other personal reasons, informal discussion with the President and their Union Coordinator should take place.
    \begin{enumerate}[i.]
        \item The member may want to step back from the role temporarily or other mediatory measures could be put in place to help the individual back into fulfilling their role.
    \end{enumerate}
\end{enumerate}